% señales.tex
%
% Copyright (C) 2024 José A. Navarro Ramón <janr.devel@gmail.com>
% 1) Código LuaLatex:
%    Licencia GPL-2.
% 2) Producto en pdf, postscript, etc.:
%    Licencia Creative Commons Recognition Share alike. (CC-BY-SA)

\section{Señales}
Las señales son un medio de comunicación entre objetos, y se emiten cuando ocurre o se completa algo.

Pasos para programar una señal:
\begin{enumerate}
  \tightlist
\item Registrar la señal. Una señal pertenece a un objeto, por lo que el registro se lleva a cabo en la
  función de inicialización de la clase del objeto.
\item Escribir un manejador de señal (\emph{signal handler}). Un manejador de señal es una función
  que se invoca cuando se emite cierta señal.
\item Conectar la señal y el manejador. Las señales se conectan a los manejadores mediante
  \texttt{g\_connect\_signal} u otra perteneciente a la misma familia de funciones.
  \item Emitir la señal.
  \end{enumerate}

  Los pasos uno y cuatro se realizan sobre el objeto al que pertenece la señal. El tercero se lleva a cabo
  generalmente fuera del objeto.

  El uso de las señales es complicado y lleva mucho tiempo explicar todas sus características. El contenido
  de esta sección se limita al mínimo necesario para escribir una señal sencilla, y por lo tanto no es del todo
  preciso. Si se necesitara una información más precisa, se puede consultar la guía de referencia de la
  API de \textsf{GObject}. Hay cuatro partes de la guía que describen las señales:
  \begin{itemize}
    \tightlist
  \item \href{https://docs.gtk.org/gobject/concepts.html#signals}
    {\emph{\textsf{Type System Concepts - signals}}}.
  \item \href{https://docs.gtk.org/gobject/#functions}
    {\textsf{Functions (g\_signal\_xxx  series)}}.
  \item   \href{https://docs.gtk.org/gobject/#function_macros}
    {\emph{\textsf{Function Macros (g\_signal\_xxx series)}}}.
  \item   \href{https://docs.gtk.org/gobject/tutorial.html#how-to-create-and-use-signals}
    {\emph{\textsf{GObject Tutorial -- How to create and use signals}}}.
\end{itemize}

\subsection{Registro de señales}
El ejemplo que veremos aquí es el de una señal que se emite cuando ocurre una división entre cero
(\emph{division-by-zero}). Primero, debemos asignar un nombre a la señal. El nombre de una señal consta
de letras, dígitos, guiones (-) y subrayados (\_). El primer carácter del nombre debe ser una letra. Elegiremos
pues, la cadena \texttt{"div-by-zero"} como nombre de nuestra señal.

Existen cuatro funciones para registrar una señal. Aquí usaremos \texttt{g\_signal\_new} para la nuestra.

\begin{lstlisting}[language=C, numbers=none]
guint
g_signal_new (const gchar *signal_name,
              GType itype,
              GSignalFlags signal_flags,
              guint class_offset,
              GSignalAccumulator accumulator,
              gpointer accu_data,
              GSignalCMarshaller c_marshaller,
              GType return_type,
              guint n_params,
              ...);
\end{lstlisting}

Se necesita mucho tiempo para explicar cada parámetro. Por ahora, se mostrará la función
\texttt{g\_signal\_new} tal y como se utilizará en nuestro fichero \textsf{tdouble.c}. Los comentarios
de los códigos que siguen se mantendrán en inglés, para que la traducción no desvirtúe su significado
correcto:
\begin{lstlisting}[language=C, numbers=none]
t_double_signal =
g_signal_new ("div-by-zero",
              G_TYPE_FROM_CLASS (class),
              G_SIGNAL_RUN_LAST | G_SIGNAL_NO_RECURSE | G_SIGNAL_NO_HOOKS,
              0 /* class offset.Subclass cannot override the class handler (default handler). */,
              NULL /* accumulator */,
              NULL /* accumulator data */,
              NULL /* C marshaller. g_cclosure_marshal_generic() will be used */,
              G_TYPE_NONE /* return_type */,
              0     /* n_params */
              );
\end{lstlisting}

\begin{itemize}
  \tightlist
\item \texttt{t\_double\_signal} es una variable estática de tipo \texttt{guint}, que es equivalente a
  \texttt{unsigned int}. Cuando se define una señal, la función \texttt{g\_signal\_new} retorna este tipo para la
  variable que almacena la señal.
\item El segundo parámetro es el tipo \texttt{GType} del objeto a la que pertenece la señal.
  \texttt{G\_TYPE\_FROM\_CLASS (class)} retorna el tipo que corresponde a la clase (\texttt{class} es un
  puntero a la clase del objeto).
\item El tercer parámetro es una bandera (\emph{flag}) de la señal. Se necesitan muchas páginas para
  explicar este parámetro. Así que por ahora, lo dejaremos tal y como está. Tal y como está este argumento,
  se puede utilizar en muchos casos. En el enlace:
  \href{https://docs.gtk.org/gobject/flags.SignalFlags.html}
  {\emph{\textsf{GObject API Reference  - SignalFlags}}},
  se pueden consultar los \textsf{SignalFlags}.
\item El octavo parámetro es el tipo \textsf{GType} de retorno del manejador, que en este caso es
  \texttt{G\_TYPE\_NONE}, que significa que el manejador no retorna ningún valor cuando se ejecuta.
\item El noveno parámetro (\texttt{n\_params}) es el número de parámetros de la señal. La que
  estamos considerando aquí (\texttt{"div-by-zero"}) no tiene parámetros, por tanto ponemos un cero.
\item Como se afirmó anteriormente, esta función se debe poner en la función de inicialización de la
  clase (\texttt{t\_double\_class\_init}).
\item Se pueden utilizar otras funciones, como \texttt{g\_signal\_newv}. Una descripción de esta
  última se puede ver en
  \href{https://docs.gtk.org/gobject/func.signal_newv.html}
  {\emph{\textsf{GObject API Reference - signal\_newv}}}.      
\end{itemize}

\subsection{Manejador de señal}
Un manejador de señal es una función que se ejecuta cuando se emite una cierta señal. Los manejadores
tienen dos parámetros:
\begin{itemize}
  \tightlist
\item La instancia a la que pertenece la señal.
\item Un puntero a unos datos de usuario (\emph{user data}) que se proporcionan cuando se conecta
  la señal al manejador. La señal \textsf{"div-by-zero"} no necesita datos de usuario.'
\end{itemize}

Para nuestro caso, el manejador sería algo así:
\begin{lstlisting}[language=C, numbers=none]
  void div_by_zero_cb (TDouble *self, gpointer user_data) { ... ... ...}
\end{lstlisting}
El primer argumento \texttt{self} es la instancia sobre la que se emite la señal. El segundo parámetro se
puede obviar:
\begin{lstlisting}[language=C, numbers=none]
  void div_by_zero_cb (TDouble *self) { ... ... ...}
\end{lstlisting}

Si una señal tiene parámetros, estos se insertan entre la instancia y los datos de usuario. Por ejemplo, el
manejador de la señal \textsf{"window-added"} en la instancia \textsf{GtkApplication} es:
\begin{lstlisting}[language=C, numbers=none]
  void window_added (GtkApplication* self, GtkWindow* window, gpointer user_data);
\end{lstlisting}
El segundo argumento \texttt{window} es el parámetro de la señal. La señal \textsf{"window-added"} se
emite cuando se añade una nueva ventana a la aplicación.
El parámetro \texttt{window} es un puntero a la ventana añadida. Ver la referencia a la API de GTK
para más información:
  \href{https://docs.gtk.org/gtk4/signal.Application.window-added.html}
  {\emph{\textsf{Gtk API Reference - window-added}}}.      

En nuestro caso, el manejador de \textsf{"div-by-zero"} mostrará un mensaje de error.
\begin{lstlisting}[language=C, numbers=none]
  static void
  div_by_zero_cb (TDouble *self, gpointer user_data) {
    g_print ("\nError: division by zero.\n\n");
  }
\end{lstlisting}

\subsection{Conexión de la señal}
Una señal y un manejador se conectan mediante la función \texttt{g\_signal\_connect}:
\begin{lstlisting}[language=C, numbers=none]
  g_signal_connect (self, "div-by-zero", G_CALLBACK (div_by_zero_cb), NULL);
\end{lstlisting}

\begin{itemize}
  \tightlist
\item El primer argumento, \texttt{self}, es la instancia a la que se conecta la señal.
\item El segundo argumento es el nombre de la señal.
\item El tercer argumento es el manejador de la señal, se debe utilizar con la macro
  \texttt{G\_CALLBACK} que ajusta el tipo de la función.
\item El último argumento representa los datos de usuario. Como la señal no los necesita,
  se asigna el valor \texttt{NULL}.
\end{itemize}

\subsection{Emisión de la señal}
La señales se emiten hacia el objeto u objetos asociados. El siguiente código se utilizará en
el próximo ejemplo de aplicación que usa el tipo \textsf{TDouble}:
\begin{lstlisting}[language=C, numbers=none]
  TDouble *
  t_double_div (TDouble *self, TDouble *other) {
    ... ... ...
    if ((! t_double_get_value (other, &value)))
    return NULL;
    else if (value == 0) {
      g_signal_emit (self, t_double_signal, 0);
      return NULL;
    }
    return t_double_new (self->value / value);
  }
\end{lstlisting}

\begin{itemize}
  \tightlist
\item El primer parámetro es la instancia que emite la señal.
\item El segundo parámetro es el identificador de la señal (\textsf{signal id}), que es el valor que
  retorna la función \texttt{g\_signal\_new}.
\item El tercer parámetro es un ``detalle'' (\textsf{detail}). \textsf{"div-by-zero"} no tiene ningún ``detalle'',
  por lo que el argumento es cero. Los ``detalles'' no se explican en esta sección, aunque generalmente se
  suele poner un cero como tercer argumento. si quieres saber algo sobre los ``detalles'', ver
  \href{https://docs.gtk.org/gobject/concepts.html#the-detail-argument}
  {\emph{\textsf{GObject API Reference - Signal Detail}}}.
  \item Cuando una señal tiene parámetros, estos se insertan a partir del cuarto argumento.
\end{itemize}





  
%%% Local Variables:
%%% coding: utf-8
%%% mode: latex
%%% TeX-engine: luatex
%%% TeX-master: "../GObject_tutorial_es.tex"
%%% End:

% LaTeX-command: "lualatex --shell-escape"
