% introduccion.tex
%
% Copyright (C) 2024 José A. Navarro Ramón <janr.devel@gmail.com>
% 1) Código LuaLatex:
%    Licencia GPL-2.
% 2) Producto en pdf, postscript, etc.:
%    Licencia Creative Commons Recognition Share alike. (CC-BY-SA)

% OBSERVACIÓN:
% Una función se puede escribir así, utilizando el paquete 'listings':
% \passthrough{\lstinline!g\_object\_new!}
% El tamaño de la fuente es algo menor que el que se obtiene mediante
% \texttt{g\_object\_new}. Aquí sigo este segundo método.

% Web:
%https://github.com/ToshioCP/Gobject-tutorial/blob/main/gfm/sec3.md

\section{Introducción a \textsf{GObject}}

\subsection{Motivación} \label{subsec:motivacion}
Desde hace mucho, he tenido  interés en programar aplicaciones con la API de \textsf{GTK}.
Según la Wikipedia, \textsf{GTK} es una biblioteca de componentes gráficos multiplataforma para desarrollar
interfaces gráficas de usuario, que está licenciada bajo los términos de la GNU LGPL, por lo que permite la creación de software libre e incluso privativo.
Pero, tenía otras tareas que no me permitían estudiar con profundidad dicha API, y además, cuando encontraba información, me parecía un galimatías que no llegaba a entender, si acaso un poco.
Como mucho, podía programar algo, pero copiando el código que encontraba en los textos, pero no terminaba de tener una imagen global.
Una parte del problema tenía que ver con el sistema de objetos \textsf{GObject} en el que se basa \textsf{GTK}.

Ahora dispongo de más tiempo, no todo el que quisiera para dedicar al sistema, por lo que escribo este tutorial  para consolidar lo que voy aprendiendo.
Debo reconocer la ayuda que he encontrado en los textos y tutoriales que me han ayudado mucho y a los que hago referencia al final del libro. Hay uno en particular [CTLP] que me ha servido para estructurar el tutorial, al menos el principio. No obstante, aquí y allá he añadido o eliminado algo, para adaptarme a mi manera de entender la materia.

\subsection{Objetivos} \label{subsec:objetivos}
A continuación escribo algunos posibles objetivos. El primero es el que estoy
intentando terminar en este texto. El resto los intentaré alcanzar, aunque puede
que sean demasiados para mí.
\begin{enumerate}
  \tightlist
\item Escribir en castellano un tutorial de \textsf{GObject} mezclando ideas de otros
  tutoriales y de documentación, de manera que pueda servir a cualquiera que esté
  interesado en la materia.
\item Utilizar \textsf{GObject} y/o \textsf{GTK} en \text{C}, \text{Julia}  y en
  \textsf{Rust}.
\item Utilizar diversos programas para construir ejecutables con \textsf{GObject},
  principalmente, \textsf{make}, \textsf{meson}, \textsf{cmake} y \textsf{ninja}.
\item Implementar GTK para Common Lisp.
\end{enumerate}

\subsection{¿Qué es {\sffamily GObject}?} \label{subsec:introduccion_gobject}
\textsf{GObject} es el sistema de objetos en el que descansa \textsf{GTK}.
Proporciona la clase base, a partir de la que se construye el resto de los elementos de
\textsf{GTK}.
Por ejemplo,  \textsf{GtkApplicationWindow}, que representa la ventana principal de una
aplicación \textsf{GTK}, deriva en última instancia de \textsf{GObject}, como se puede apreciar en la figura~\ref{fig:jerarquia-gtkappwindow}.
\begin{figure}[ht]
  \centering
  \def\scl{1}
  \begin{tikzpicture}[%
    scale=\scl,
    every node/.style={font=\sffamily, outer xsep = -2pt},
    background/.style={
      line width=\bgborderwidth,
      draw=\bgbordercolor,
      fill=\bgcolor,
    },
    ]
    % COORDENADAS
    % - 
    % DIBUJO
    % Ejemplo de jerarquía de objetos de 'GObject'
    \node (gobject) {GObject};
    \node [below right = 4ex and 4.5em of gobject.west, anchor=west] (ginitiallyunowned)
    {GInitiallyUnowned};
    \node [below right = 4ex and 4.5em of ginitiallyunowned.west, anchor = west]
    (gtkwidget)  {GtkWidget};
    \node [below right = 4ex and 4.5em of gtkwidget.west, anchor = west] (gtkwindow)
    {GtkWindow};
    \node [below right = 4ex and 4.5em of gtkwindow.west, anchor=west] (gtkappwindow)
    {GtkApplicationWindow};
    \draw [shorten <= 0.5ex] (gobject.base west) + (1.7em,0) |- (ginitiallyunowned.west);
    \draw [shorten <= 0.5ex] (ginitiallyunowned.base west) + (1.7em,0) |-
    (gtkwidget.west);
    \draw [shorten <= 0.5ex] (gtkwidget.base west) + (1.7em,0) |- (gtkwindow.west);
    \draw [shorten <= 0.5ex] (gtkwindow.base west) + (1.7em,0) |- (gtkappwindow.west);
    % Fondo amarillo
    \coordinate (limsupizda) at ($(gobject.north west)+(-6pt,6pt)$);
    \coordinate (liminfdcha) at ($(gtkappwindow.south east)+(6pt,-6pt)$);
    \begin{scope}[on background layer]
      \node [background, fit= (liminfdcha) (limsupizda)] {};
    \end{scope}
  \end{tikzpicture}
  \caption{Jerarquía del objeto {\sffamily GtkAplicationWindow} de GTK.}
  \label{fig:jerarquia-gtkappwindow}
\end{figure}

{\sffamily GObject} también se puede utilizar para desarrollar una jerarquía de objetos independiente de los elementos de GTK y que sea completamente independiente de ellos.

\subsection{Clase e instancias de \textsf{GObject}} \label{subsec:clase_instancias_gobject}

\subsubsection{Teoría}
En sentido amplio, \textsf{GObject} es un sistema de objetos que proporciona dos entidades relacionadas: instancias y clases. La función \texttt{g\_object\_new} se
encarga de ello:
\begin{itemize}
  \tightlist
\item Se crea la clase \textsf{GObject} ---\textsf{GObjectClass}--- la primera vez que
  se ejecuta la función \texttt{g\_object\_new}. Solo puede existir una clase que
  represente a \textsf{GObject} que se crea la primera vez que se ejecuta la función.
  La clase se materializa como una estructura en lenguaje C, pero como es compleja no es
  conveniente mostrarla aquí. Solo decir que la primera vez que se llama a \texttt{g\_object\_new} se reserva espacio para la estructura \textsf{GObjectClass} y se
  inicializa.
\item Se crea una instancia de \textsf{GObject} cada vez que se ejecuta la función
  \texttt{g\_object\_new}. Puede haber una o más instancias de \textsf{GObject}.
  A continuación se muestra la estructura que las representa,tomada de \textsf{gobject.h}':
\begin{lstlisting}[language=C]
  typedef struct _GObject GObject;
  struct _GObject {
    GTypeInstance g_type_instance;

    /*< private >*/
    guint ref_count; /* (atomic) */
    GData *qdata;
  };
\end{lstlisting}
La función \passthrough{\lstinline!g\_object\_new!} reserva espacio para la estructura,
inicializa la memoria y devuelve el puntero a esa estructura. La memoria representada por este puntero es una instancia de \textsf{GObject}.
\end{itemize}

\subsubsection{Ejemplo 1}
En este ejemplo se crean dos instancias diferentes de la clase \textsf{GObject}, comprobándose que estas pertenecen a la misma clase. En el directorio '\textsf{prog}' se
encuentran este y otros programas de ejemplo. Puede consultar las instrucciones para
compilar los programas en la sección \ref{sec:compilacion}.

El listado es:
\begin{lstlisting}[language=C, numbers=left]
/*
 * ejemplo01.c
 * Ejemplo de creación y destrucción de instancias de 'GObjectClass'.
 * https://github.com/ToshioCP/Gobject-tutorial/blob/main/gfm/sec2.md
 * https://github.com/ToshioCP/Gobject-tutorial/tree/main
 */

  #include <locale.h>
  #include <glib-object.h>

  int
  main (int argc, char **argv)
  {
    GObject *instancia1, *instancia2;
    GObjectClass *clase1, *clase2;

    /* Para que g_print no transforme la codificación de caracteres
    por su cuenta y no se vean los acentos */
    setlocale(LC_CTYPE, "");

    /* Creación de dos instancias de la clase 'GObject' */
    instancia1 = g_object_new (G_TYPE_OBJECT, NULL);
    instancia2 = g_object_new (G_TYPE_OBJECT, NULL);
    g_print ("Dirección de 'instancia1': %p\n", instancia1);
    g_print ("Dirección de 'instancia2': %p\n", instancia2);

    /* Comprobación de que las dos instancias pertenecen a la misma clase */
    clase1 = G_OBJECT_GET_CLASS (instancia1);
    clase2 = G_OBJECT_GET_CLASS (instancia2);
    g_print ("Dirección de la clase de 'instancia1': %p\n", clase1);
    g_print ("Dirección de la clase de 'instancia2': %p\n", clase2);

    /* Destrucción de las instancias */
    g_object_unref(instancia1);
    g_object_unref(instancia2);

    return 0;
  }
\end{lstlisting}

\subsubsection{Salida del programa 'ejemplo01'}
\textsf{Dirección de 'instancia1': 0x5605fdedb800}\par
\textsf{Dirección de 'instancia2': 0x5605fdedb820}\par
\textsf{Dirección de la clase de 'instancia1':  0x5605fdedb4f0}\par
\textsf{Dirección de la clase de 'instancia2':  0x5605fdedb4f0}\par


\subsubsection{Comentarios}
\begin{itemize}
\item Líneas 14 y 15:\par
  Se declaran dos punteros a objetos de \textsf{GObject}, por lo que podremos crear dos
  instancias de esta clase.
  Se declaran dos punteros de \textsf{GObjectClass}, a los que asignaremos la clase a la
  que pertenecen las instancias anteriores.
\item Línea 19:\par
  Se asigna a la variable de entorno \texttt{L\_CTYPE} la cadena vacía para que la
  función \passthrough{\lstinline!g\_print!} no modifique la codificación que se utiliza
  (en este caso UTF-8) y se puedan ver los acentos correctamente en la consola UTF-8.
\item Líneas 22 -- 25:\par
  Se crean dos instancias diferentes de \textsf{GObject} mediante la función
  \passthrough{\lstinline!g\_object\_new!}. Posteriormente se comprueba que son
  diferentes, dado que sus punteros son distintos.
\item Líneas 28 -- 31:\par
  Se obtiene el puntero a la clase a la que pertenecen 'instancia1' e 'instancia2',
  estos punteros se almacenan en 'clase1' y 'clase2', respectivamente. Esta tarea se
  realiza mediante la macro \passthrough{\lstinline!G\_OBJECT\_GET\_CLASS!}. Al imprimir
  en pantalla los punteros de las dos clases observamos que son iguales, de manera que
  ambas instancias pertenecen a la misma clase \textsf{GObjectClass}, que se creó la
  primera vez que se ejecutó la función \passthrough{\lstinline!g\_object\_new!} en la
  línea 14, a la vez que se creó la primera instancia, 'instancia1'.
\item Líneas 34 -- 35:\par
  Se liberan los objetos 'instancia1' e 'instancia2' antes de que termine el programa.
  Nos podríamos preguntar qué le ocurre a la clase \textsf{GObjectClass} que se creó en la línea 14. La respuesta a se aclarará en la siguiente sección.
\end{itemize}

\begin{figure}[ht]
  \centering
  \def\scl{1}
  \begin{tikzpicture}[%
    scale=\scl,
    every node/.style={draw, font=\sffamily},
    clase/.style={minimum size=20ex, inner sep=1em, fill=green!65},
    instancia/.style={minimum size=15ex, inner sep=1em, fill=yellow!65},
    flecha/.style={line width=1.6pt, -Stealth[round]},
    background/.style={
      line width=\bgborderwidth,
      draw=\bgbordercolor,
      fill=\bgcolor,
    },
    ]
   % COORDENADAS
   % - 
   % DIBUJO
   % Clase GObject
    \node[clase, text centered]
    (gobjectclass) {\large \begin{tabular}{c} Clase GObject \\ (0x5605fdedb4f0) \end{tabular}};
   % Instancias de GObject
   \node [instancia, above right = 4ex and 10em of gobjectclass.east]
   (instancia1) {\normalsize \begin{tabular}{c} Instancia 1 \\ (0x5605fdedb800) \end{tabular}};
   \node [instancia, below right = 4ex and 10em of gobjectclass.east]
   (instancia2)  {\normalsize \begin{tabular}{c} Instancia 2 \\ (0x5605fdedb820) \end{tabular}};
   % Flechas
   %\draw[flecha] (instancia1.west) -- (gobjectclass.15);
   \draw[flecha] (instancia1.west) .. controls (3.5, 1.4) and (2.8, 0.5)  .. (gobjectclass.12);
   \draw[flecha] (instancia2.west) .. controls (3.5, -1.4) and (2.8, -0.5)  .. (gobjectclass.-12);

   % Puntos de control de las flechas
%   \filldraw[fill=red, draw=black] (3.5, 1.4) circle[radius=2pt];
%   \filldraw[fill=red, draw=black] (2.8, 0.5) circle[radius=2pt];
%   
%   \filldraw[fill=green, draw=black] (3.5, -1.4) circle[radius=2pt];
%   \filldraw[fill=green, draw=black] (2.8, -0.5) circle[radius=2pt];
   
%   Fondo amarillo
   \def\extra{15}
   \coordinate (limsupdcha) at ($(instancia1.north east) + (\extra pt, \extra pt)$);
   \coordinate (liminfdcha) at ($(instancia2.south east) - (\extra pt, \extra pt)$);
   \coordinate(limizda) at ($(gobjectclass.west) - (\extra pt,0)$);
   \begin{scope}[on background layer]
     \node [background, fit= (limsupdcha) (liminfdcha) (limizda)] {};
   \end{scope}
\end{tikzpicture}
\caption{Instancias de la clase \texttt{GObjectClass} creadas en el ejemplo 1.
  Los valores hexadecimales entre paréntesis son posibles direcciones de la clase
  \textsf{GObject} (\texttt{GObjectClass}) y de las dos instancias creadas de esa clase.}
\label{fig:gobject-instancias-ejemplo-uno}
\end{figure}


\subsection{Referencias a instancias de \textsf{GObject}} \label{subsec:ref_instancias_gobject}
\subsubsection{Teoría}
Toda instancia de \textsf{GObject} contiene un bloque de memoria asociado que el sistema asigna e inicializa cuando esta se crea.
Cuando la instancia deja de ser necesaria, se debe eliminar liberando la memoria que
tiene asignada.
% \href{run:./sample.txt}{here}

¿Pero cómo sabe el sistema \textsf{GObject}  que no se utiliza una instancia determinada?
Lo resuelve mediante el
\href{https://es.wikipedia.org/wiki/Conteo_de_referencias}{conteo de referencias}
(\href{https://en.wikipedia.org/wiki/Reference_counting}{\textit{reference counting}}).
Cuando se crea una instancia (mediante la función \passthrough{\lstinline!g\_object\_new!}), se dice que la instancia está referida
(\emph{referred}), y se le asigna una cuenta de referencia (\textit{reference count}), haciendo la variable \texttt{ref\_count} igual a 1, indicando que hay un módulo, función u objeto que la está utilizando. Llamemos \textbf{A} a este elemento.
Ahora pueden ocurrir diversas circunstancias, entre otras:
\begin{itemize}
  \tightlist
\item Si \textbf{A} no necesita utilizar más la instancia, la desreferencia mediante
  \passthrough{\lstinline!g\_object\_unref!}, utilizando un puntero a la instancia como
  argumento de la función.
  Entonces, el sistema decrementa la cuenta de referencia en una unidad, de manera que
  ahora valdría cero.
  En este supuesto, el sistema detecta que ya no se utiliza la instancia y la elimina,
  liberando su memoria asociada.
\item Pero imaginemos que se pasa la instancia (su puntero) a otro módulo, función u
  objeto del programa, que llamaremos \textbf{B}. Este debe referenciar la instancia
  para evitar la posibilidad de que \textbf{A} la elimine antes de \textbf{B} termine de
  utilizarla, y para ello debe llamar a la función    \passthrough{\lstinline!g\_object\_ref!}, utilizando la el puntero a la instancia como
  argumento.
  Ahora la cuenta de referencia de la instancia se incrementaría en una unidad pasando a
  valer 2.

  Si \textbf{A}, decidiera que ya no necesita más la instancia, llamaría a
  \passthrough{\lstinline!g\_object\_unref!} y la cuenta de referencia bajaría a 1. Este
  valor indicaría al sistema que todavía queda otro elemento que la esta utilizando (en
  este caso \textbf{B}). Así, la memoria asociada no se liberaría, hasta que \textbf{B}
  decidiera prescindir de la instania, en cuyo caso, la cuenta de referencia bajaría a
  cero y, solo entonces, se liberaría la memoria asociada y la instancia dejaría de
  existir en el sistema.
\end{itemize}

\subsubsection{Ejemplo 2}
En el siguiente ejemplo se una instancia de la clase \textsf{GObject}, después se
referencia una vez.
Finalmente se desreferencia dos veces hasta que el sistema destruye la instancia y
libera la memoria.
Ver instrucciones para compilar los programas en la sección \ref{sec:compilacion}.

El listado es:
\begin{lstlisting}[language=C, numbers=left]
  /*
  * ejemplo02.c
  * Prueba de las referencias a una instancia.
  */

  #include <locale.h>
  #include <glib-object.h>

  static void
  show_ref_count  (GObject *inst)
  {
    if  (G_IS_OBJECT (inst)) {
      /* Los usuarios no deberían utilizar el método 'ref_count' en sus programas.
      pues interferirían de una forma impredecible con GObject */
      /* Aquí se utiliza solo como demostración */
      g_print ("  Cuenta de referencia de la instancia: %d.\n", inst->ref_count);
    }
    else {
      g_print ("  La instancia recibida no es un objeto de 'GObjectClass'.\n");
    }
  }

  int
  main (int argc, char **argv)
  {
    GObject *instancia;

    // Para que g_print no transforme la codificación de caracteres
    // por su cuenta y no se vean los acentos.
    setlocale(LC_CTYPE, "");

    g_print ("- Se crea el objeto 'instancia', llamando a 'g_object_new'.\n");
    instancia = g_object_new(G_TYPE_OBJECT, NULL);
    show_ref_count (instancia);
    g_print("\n");
  
    g_print ("- Se añade una referencia a 'instancia', llamando a 'g_object_ref'.\n");
    g_object_ref (instancia);
    show_ref_count (instancia);
    g_print("\n");

    g_print ("- Se elimina una referencia, llamando a 'g_object_unref'.\n");
    g_object_unref(instancia);
    show_ref_count (instancia);
    g_print("\n");

    g_print ("- Se elimina la última referencia, llamando a 'g_object_unref'.\n");
    g_object_unref (instancia);
    g_print ("  La cuenta de referencia vale cero y la instancia se destruye.\n");
    g_print ("  Su memoria asociada se libera.\n");
    g_print("  Una llamada a 'show_ref_count(instance)' daría un error, pues la\n");
    g_print("  instancia  no existe.\n");
    g_print("  Por ello, se ha comentado esta línea en el código fuente.\n");
    /* show_ref_count (instancia); */
  
    return 0;
  }
\end{lstlisting}

\subsubsection{Salida del programa 'ejemplo02'}
\textsf{- Se crea el objeto 'instancia', llamando a 'g\_object\_new'.}\par
\textsf{  La cuenta de referencia de la instancia es 1.}
\par\textsf{ }\par
\textsf{- Se añade una referencia a 'instancia', llamando a 'g\_object\_ref'.}\par
\textsf{  La cuenta de referencia de la instancia es 2.}\par
\par\textsf{ }\par
\textsf{- Se elimina una referencia a 'instancia', llamando a 'g\_object\_unref'.}\par
\textsf{  La cuenta de referencia de la instancia es 1.}\par
\par\textsf{ }\par
\textsf{- Se elimina la última referencia a 'instancia', llamando a 'g\_object\_unref'.}\par
\textsf{  Ahora, la cuenta de referencia vale cero y la instancia se destruye.}\par
\textsf{  Su memoria asociada se libera.}\par
\textsf{  Una llamada a 'show\_ref\_count(instancia) daría un error, pues la}\par
\textsf{  instancia ya no existe.}\par
\textsf{  Por ello, se ha comentado esta línea en el código fuente.}\par

\subsubsection{Comentarios}
\begin{itemize}
\item Línea 9:\par
  Se declara la función \passthrough{\lstinline!show\_ref\_count!} como \texttt{static
    void}.
  El identificador \texttt{static} especifica que el alcance (\textit{scope}) de la
  función es
  este fichero. Si hubiera algún otro fichero, no podría utilizar la función que se
  define aquí, pero podría llamar a otra definida en otro lugar que se llamara igual.
\item Línea 26:\par
  Se declara un puntero a objeto de \textsf{GObject}, lo que no permitirá crear una
  instancia.
\item Línea 30:\par
  Se asigna a la variable de entorno \texttt{L\_CTYPE} la cadena vacía para que la
  función
  \passthrough{\lstinline!g\_print!} no modifique la codificación que se utiliza (en
  este caso UTF-8)
  y se puedan ver los acentos correctamente en la consola UTF-8.
\item Líneas 32 -- 35:\par
  Se crea una instancia de \textsf{GObject} mediante la función \passthrough{\lstinline!g\_object\_new!}.
  Se muestra el valor, 1,  de la cuenta de referencia de esa instancia.
\item Líneas 37 -- 40:\par
  Se añade una referencia a la instancia mediante la función  \passthrough{\lstinline!g\_object\_ref!}.
  Se muestra el valor, 2,  de la cuenta de referencia de esa instancia.
\item Líneas 42 -- 45:\par
  Se elimina una referencia a la instancia mediante la función  \passthrough{\lstinline!g\_object\_unref!}.
  Se muestra el valor, 1,  de la cuenta de referencia de esa instancia.
\item Líneas 47 -- 54:\par
  Se elimina la  última referencia que quedaba de la instancia.
  Por tanto, se destruye el objeto 'instancia' y se libera su memoria asociada.
  Por esa razón, no se puede mostrar su cuenta de referencia, porque
  \passthrough{\lstinline!instancia->ref_count!}, de la línea 16 daría un error.
  Por ello, se ha comentado la línea 54 que produciría ese error.
  El lector puede descomentar la citada línea y comprobar el error de segmentación que
  se produciría.
\end{itemize}


%%% Local Variables:
%%% coding: utf-8
%%% mode: latex
%%% TeX-engine: luatex
%%% TeX-master: "../GObject_tutorial_es.tex"
%%% End:

