% introduccion.tex
%
% Copyright (C) 2024 José A. Navarro Ramón <janr.devel@gmail.com>
% 1) Código LuaLatex:
%    Licencia GPL-2.
% 2) Producto en pdf, postscript, etc.:
%    Licencia Creative Commons Recognition Share alike. (CC-BY-SA)

\section{Introducción}

\subsection{Motivación} \label{subsec:motivacion}
Desde hace mucho tiempo he tenido interés en programar aplicaciones con la API de GTK.
Según la Wikipedia, GTK es una biblioteca de componentes gráficos multiplataforma para desarrollar
interfaces gráficas de usuario, que está licenciada bajo los términos de la GNU LGPL, por lo que
permite la creación de software libre e incluso privativo.
Por una parte, tenía otras tareas que no me dejaban centrarme en este \textit{framework},
por otra, cuando intentaba acercarme a los textos para principiantes que encontraba, me parecía un
galimatías que no llegaba a entender, si acaso un poco, de manera que sí podía programar algo, pero
copiando lo que encontraba en los textos y no terminaba de avanzar.

Gran parte del problema tenía que ver con el sistema de objetos \textsf{GObject} en el que se basa GTK.
Ahora dispongo de más tiempo para aprender a programar con este sistema. Por tanto, escribo este
resumen para consolidar  lo que voy aprendiendo. Además, es posible que sirva a algunas otras personas.

Debo reconocer la ayuda que he encontrado en los textos y tutoriales que me han ayudado mucho y
a los que hago referencia al final del texto. Hay uno en particular [CTLP] del que he tomado casi todo lo
que incluyo. No obstante, aquí y allá he añadido o eliminado algo, para adaptarme a mi manera de entender
la materia.
\subsection{Objetivos} \label{subsec:objetivos}
En orden:
\begin{enumerate}
  \tightlist
\item Utilizar \textsf{GObject} en los programas en C.
\item Desarrollar aplicaciones con GTK en C.
\item Escribir en castellano un tutorial de GObject mezclando ideas de otros tutoriales y de documentación.
\item Implementar GTK en Common Lisp.
\end{enumerate}

\subsection{Introducción a {\sffamily GObject}} \label{subsec:introduccion_gobject}
\textsf{GObject} es el sistema de objetos en el que descansa GTK. Además es el objeto, a partir del que se
construye el resto que forman los elementos de GTK. Por ejemplo, el objeto \textsf{GtkApplicationWindow},
que representa la ventana principal de una aplicación GTK, deriva en última instancia de \textsf{GObject},
como se puede apreciar en la figura~\ref{fig:jerarquia-gtkappwindow}.
\begin{figure}[ht]
  \centering
  \def\scl{1}
  \begin{tikzpicture}[%
    scale=\scl,
    every node/.style={font=\sffamily, outer xsep = -2pt},
     background/.style={
      line width=\bgborderwidth,
      draw=\bgbordercolor,
      fill=\bgcolor,
    },
   ]
   % COORDENADAS
   % - 
   % DIBUJO
   % Ejemplo de jerarquía de objetos de 'GObject'
   \node (gobject) {GObject};
   \node [below right = 4ex and 4.5em of gobject.west, anchor=west] (ginitiallyunowned) {GInitiallyUnowned};
   \node [below right = 4ex and 4.5em of ginitiallyunowned.west, anchor = west] (gtkwidget)  {GtkWidget};
   \node [below right = 4ex and 4.5em of gtkwidget.west, anchor = west] (gtkwindow) {GtkWindow};
   \node [below right = 4ex and 4.5em of gtkwindow.west, anchor=west] (gtkappwindow) {GtkApplicationWindow};
   \draw [shorten <= 0.5ex] (gobject.base west) + (1.7em,0) |- (ginitiallyunowned.west);
   \draw [shorten <= 0.5ex] (ginitiallyunowned.base west) + (1.7em,0) |- (gtkwidget.west);
   \draw [shorten <= 0.5ex] (gtkwidget.base west) + (1.7em,0) |- (gtkwindow.west);
   \draw [shorten <= 0.5ex] (gtkwindow.base west) + (1.7em,0) |- (gtkappwindow.west);
   % Fondo amarillo
   \coordinate (limsupizda) at ($(gobject.north west)+(-6pt,6pt)$);
   \coordinate (liminfdcha) at ($(gtkappwindow.south east)+(6pt,-6pt)$);
   \begin{scope}[on background layer]
     \node [background, fit= (liminfdcha) (limsupizda)] {};
   \end{scope}
\end{tikzpicture}
\caption{Jerarquía del objeto {\sffamily GtkAplicationWindow} de GTK.}
\label{fig:jerarquia-gtkappwindow}
\end{figure}

{\sffamily GObject} también se puede utilizar para desarrollar una jerarquía de objetos independiente de los
elementos de GTK y que sea completamente independiente de ellos.


\subsection{Creación de instancias de \textsf{GObject}} \label{subsec:creacion_instancias_gobject}

\subsubsection{Teoría}
Para crear una instancia de \textsf{GObject} se utiliza la función
\passthrough{\lstinline!g\_object\_new!}.
\textsf{GObject} proporciona dos entidades relacionadas: instancias y clases:
\begin{itemize}
  \tightlist
\item Una clase \textsf{GObject} la primera vez que se ejecuta la función
  \passthrough{\lstinline!g\_object\_new!}. Sólo puede existir una clase que represente a \textsf{GObject}.
  Por eso solo se crea la primera vez que se ejecuta la función.
\item Una instancia de la clase \textsf{GObject} cada vez que se ejecuta la función
  \passthrough{\lstinline!g\_object\_new!}. Puede haber una o más instancias de la clase \textsf{GObject}.
\end{itemize}

En sentido amplio, \textsf{GObject} es el objeto, que incluye su clase e instancias. En sentido estricto,
es la definición de una estructura en C.
\begin{lstlisting}[language=C]
  typedef struct _GObject GObject;
  struct _GObject {
    GTypeInstance g_type_instance;

    /*< private >*/
    guint ref_count; /* (atomic) */
    GData *qdata;
  };
\end{lstlisting}

La función \passthrough{\lstinline!g\_object\_new!} reserva espacio para la estructura \textsf{GObject},
inicializa la memoria y devuelve el puntero a esa estructura. La memoria representada por este puntero
es una instancia de \textsf{GObject}.

De a misma forma, la primera vez que se llama a \passthrough{\lstinline!g\_object\_new!} se reserva
espacio para la estructura \textsf{GObjectClass}, que representa la clase de \text{GObject}. A continuación
se muestra la definición de la estructura tomada de \passthrough{\lstinline!gobject.h!}, aunque no se
necesita entrar en detalle en su definición, pues es más compleja que la estructura anterior.

\subsubsection{Ejemplo 1}
En el siguiente ejemplo se crean dos instancias diferentes de la clase \textsf{GObject} y se comprueba que
solo hay una clase para estos objetos. En el directorio 'prog' se encuentran los programas de ejemplo.
Para compilarlos todos basta con hacer
\begin{lstlisting}[language=bash]
  $ cd prog
  $ make
\end{lstlisting}
o bien, solo el primer ejemplo:
\begin{lstlisting}[language=bash]
  $ cd prog
  $ make ejemplo01
\end{lstlisting}

El listado es:
\begin{lstlisting}[language=C, numbers=left]
  #include <locale.h>
  #include <glib-object.h>

  int
  main (int argc, char **argv) {
    GObject *instancia1, *instancia2;
    GObjectClass *clase1, *clase2;

    // Para que g_print no transforme la codificación de caracteres
    // por su cuenta y no se vean los acentos.
    setlocale(LC_CTYPE, "");

    // Creación de dos instancias de la clase 'GObject'
    instancia1 = g_object_new (G_TYPE_OBJECT, NULL);
    instancia2 = g_object_new (G_TYPE_OBJECT, NULL);
    g_print ("Dirección de 'instancia1': %p\n", instancia1);
    g_print ("Dirección de 'instancia2': %p\n", instancia2);

    // Comprobación de que solo hay una clase 'GObject'
    clase1 = G_OBJECT_GET_CLASS (instancia1);
    clase2 = G_OBJECT_GET_CLASS (instancia2);
    g_print ("Dirección de la clase de 'instancia1': %p\n", clase1);
    g_print ("Dirección de la clase de 'instancia2': %p\n", clase2);

    // Liberación de las instancias
    g_object_unref(instancia1);
    g_object_unref(instancia2);

    return 0;
  }
\end{lstlisting}

\subsubsection{Ejemplo de salida del programa 'ejemplo1'}
\textsf{Dirección de 'instancia1': 0x5605fdedb800}\par
\textsf{Dirección de 'instancia2': 0x5605fdedb820}\par
\textsf{Dirección de la clase de 'instancia1':  0x5605fdedb4f0}\par
\textsf{Dirección de la clase de 'instancia2':  0x5605fdedb4f0}\par


\subsubsection{Comentarios}
\begin{itemize}
\item Líneas 6 y 7\par
  Se declaran dos punteros a objetos de \textsf{GObject}, por lo que podremos crear dos
  instancias de esta clase.
  Se declaran dos punteros de \textsf{GObjectClass}, a los que asignaremos la clase a la que pertenecen
  las instancias anteriores.
\item Línea 11\par
  Se asigna a una variable de entorno la cadena vacía para que la función
  \passthrough{\lstinline!g\_print!} no modifique la codificación que se utiliza (en este caso UTF-8)
  y se puedan ver los acentos correctamente en la consola UTF-8.
\item Líneas 14 -- 17\par
  Se crean dos instancias diferentes de \textsf{GObject} mediante la función 
  \passthrough{\lstinline!g\_object\_new!}. Posteriormente se comprueba que son diferentes, dado que
  sus punteros son distintos.
\item Líneas 20 -- 23\par
  Se obtiene el puntero a la clase a la que pertenecen 'instancia1' e 'instancia2', estos punteros se
  almacenan en 'clase1' y 'clase2', respectivamente. Esta tarea se realiza mediante la macro
  \passthrough{\lstinline!G\_OBJECT\_GET\_CLASS!}. Al imprimir en pantalla los punteros de las
  dos clases observamos que son iguales, de manera que ambas instancias pertenecen a la misma
  clase \textsf{GObjectClass}, que se creó la primera vez que se ejecutó la función
  \passthrough{\lstinline!g\_object\_new!} en la línea 14, a la vez que se creó la primera instancia,
  'instancia1'.
\item Líneas 26 -- 27\par
  Se liberan los objetos 'instancia1' e 'instancia2' antes de que termine el programa. Nos podríamos
  preguntar qué le ocurre a la clase \textsf{GObjectClass} que se creó en la línea 14. La respuesta a
  esta duda se aclarará en la siguiente sección.
\end{itemize}



\begin{figure}[ht]
  \centering
  \def\scl{1}
  \begin{tikzpicture}[%
    scale=\scl,
    every node/.style={draw, font=\sffamily},
    clase/.style={minimum size=20ex, inner sep=1em, fill=green!65},
    instancia/.style={minimum size=15ex, inner sep=1em, fill=yellow!65},
    flecha/.style={line width=1.6pt, -Stealth[round]},
    background/.style={
      line width=\bgborderwidth,
      draw=\bgbordercolor,
      fill=\bgcolor,
    },
    ]
   % COORDENADAS
   % - 
   % DIBUJO
   % Clase GObject
    \node[clase, text centered]
    (gobjectclass) {\Large \begin{tabular}{c} Clase GObject \\ (0x5605fdedb4f0) \end{tabular}};
   % Instancias de GObject
   \node [instancia, above right = 4ex and 10em of gobjectclass.east]
   (instancia1) {\large \begin{tabular}{c} Instancia 1 \\ (0x5605fdedb800) \end{tabular}};
   \node [instancia, below right = 4ex and 10em of gobjectclass.east]
   (instancia2)  {\large \begin{tabular}{c} Instancia 2 \\ (0x5605fdedb820) \end{tabular}};
   % Flechas
   %\draw[flecha] (instancia1.west) -- (gobjectclass.15);
   \draw[flecha] (instancia1.west) .. controls (3.5, 1.4) and (2.8, 0.5)  .. (gobjectclass.12);
   \draw[flecha] (instancia2.west) .. controls (3.5, -1.4) and (2.8, -0.5)  .. (gobjectclass.-12);

   % Puntos de control de las flechas
%   \filldraw[fill=red, draw=black] (3.5, 1.4) circle[radius=2pt];
%   \filldraw[fill=red, draw=black] (2.8, 0.5) circle[radius=2pt];
%   
%   \filldraw[fill=green, draw=black] (3.5, -1.4) circle[radius=2pt];
%   \filldraw[fill=green, draw=black] (2.8, -0.5) circle[radius=2pt];
   
%   Fondo amarillo
   \def\extra{15}
   \coordinate (limsupdcha) at ($(instancia1.north east) + (\extra pt, \extra pt)$);
   \coordinate (liminfdcha) at ($(instancia2.south east) - (\extra pt, \extra pt)$);
   \coordinate(limizda) at ($(gobjectclass.west) - (\extra pt,0)$);
   \begin{scope}[on background layer]
     \node [background, fit= (limsupdcha) (liminfdcha) (limizda)] {};
   \end{scope}
\end{tikzpicture}
\caption{Instancias de la clase \textsf{GObjectClass} creadas en el ejemplo 1.}
\label{fig:gobject-instancias-ejemplo-uno}
\end{figure}




%%% Local Variables:
%%% coding: utf-8
%%% mode: latex
%%% TeX-engine: luatex
%%% TeX-master: "../GObject_tutorial_es.tex"
%%% End:

% LaTeX-command: "lualatex --shell-escape"
