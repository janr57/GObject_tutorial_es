% introduccion.tex
%
% Copyright (C) 2024 José A. Navarro Ramón <janr.devel@gmail.com>
% 1) Código LuaLatex:
%    Licencia GPL-2.
% 2) Producto en pdf, postscript, etc.:
%    Licencia Creative Commons Recognition Share alike. (CC-BY-SA)

\section{Introducción}

\subsection{Motivación} \label{subsec:motivacion}
Desde hace mucho tiempo he tenido interés en programar aplicaciones con la API de GTK.
Según la Wikipedia, GTK es una biblioteca de componentes gráficos multiplataforma para desarrollar
interfaces gráficas de usuario, que está licenciada bajo los términos de la GNU LGPL, por lo que
permite la creación de software libre e incluso privativo.
Por una parte, tenía otras tareas que no me dejaban centrarme en este \textit{framework},
por otra, cuando intentaba acercarme a los textos para principiantes que encontraba, me parecía un
galimatías que no llegaba a entender, si acaso un poco, de manera que sí podía programar algo, pero
copiando lo que encontraba en los textos y no terminaba de avanzar.

Gran parte del problema tenía que ver con el sistema de objetos {\sffamily GObject} en el que se basa GTK.
Ahora dispongo de más tiempo para aprender a programar con este sistema. Por tanto, escribo este
resumen para consolidar  lo que voy aprendiendo. Además, es posible que sirva a algunas otras personas.

Debo reconocer la ayuda que he encontrado en los textos y tutoriales que me han ayudado mucho y
a los que hago referencia al final del texto. Hay uno en particular [CTLP] del que he tomado casi todo lo
que incluyo. No obstante, aquí y allá he añadido o eliminado algo, para adaptarme a mi manera de entender
la materia.
\subsection{Objetivos} \label{subsec:objetivos}
En orden:
\begin{enumerate}
\item Aprender usar {\sffamily GObject}.
\item Desarrollar aplicaciones con GTK.
  \item Escribir en castellano un tutorial de GObject mezclando ideas de otros tutoriales y de documentación.
\item Implementar GTK en Common Lisp.
\end{enumerate}

\subsection{Introducción a {\sffamily GObject}} \label{subsec:introduccion_gobject}
{\sffamily GObject} es el sistema de objetos en el que descansa GTK. Además es el objeto, a partir del que se
construye el resto que forman los elementos de GTK. Por ejemplo, el objeto {\sffamily GtkApplicationWindow},
que representa la ventana principal de una aplicación GTK, deriva en última instancia de {\sffamily GObject},
como se puede apreciar en la figura~\ref{fig:jerarquia-gtkappwindow}.
\begin{figure}[ht]
  \centering
  \def\scl{1}
  \begin{tikzpicture}[%
    scale=\scl,
    every node/.style={font=\large\sffamily, outer xsep = -2pt},
     background/.style={
      line width=\bgborderwidth,
      draw=\bgbordercolor,
      fill=\bgcolor,
    },
   ]
   % COORDENADAS
   % - 
   % DIBUJO
   % Ejemplo de jerarquía de objetos de GObject 
   \node (gobject) {GObject};
   \node [below right = 4ex and 4.5em of gobject.west, anchor=west] (ginitiallyunowned) {GInitiallyUnowned};
   \node [below right = 4ex and 4.5em of ginitiallyunowned.west, anchor = west] (gtkwidget)  {GtkWidget};
   \node [below right = 4ex and 4.5em of gtkwidget.west, anchor = west] (gtkwindow) {GtkWindow};
   \node [below right = 4ex and 4.5em of gtkwindow.west, anchor=west] (gtkappwindow) {GtkApplicationWindow};
   \draw [shorten <= 0.5ex] (gobject.base west) + (1.7em,0) |- (ginitiallyunowned.west);
   \draw [shorten <= 0.5ex] (ginitiallyunowned.base west) + (1.7em,0) |- (gtkwidget.west);
   \draw [shorten <= 0.5ex] (gtkwidget.base west) + (1.7em,0) |- (gtkwindow.west);
   \draw [shorten <= 0.5ex] (gtkwindow.base west) + (1.7em,0) |- (gtkappwindow.west);
   % Fondo amarillo
   \coordinate (limsupizda) at ($(gobject.north west)+(-6pt,6pt)$);
   \coordinate (liminfdcha) at ($(gtkappwindow.south east)+(6pt,-6pt)$);
   \begin{scope}[on background layer]
     \node [background, fit= (liminfdcha) (limsupizda)] {};
   \end{scope}
\end{tikzpicture}
\caption{Jerarquía del objeto {\sffamily GtkAplicationWindow} de GTK.}
\label{fig:jerarquia-gtkappwindow}
\end{figure}

{\sffamily GObject} también se puede utilizar para desarrollar una jerarquía de objetos independiente de los
elementos de GTK y que sea completamente independiente de ellos.


%%% Local Variables:
%%% coding: utf-8
%%% mode: latex
%%% TeX-engine: luatex
%%% TeX-master: "../GObject_tutorial_es.tex"
%%% End:

% LaTeX-command: "lualatex --shell-escape"
