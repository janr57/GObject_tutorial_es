% propiedades.tex
%
% Copyright (C) 2024 José A. Navarro Ramón <janr.devel@gmail.com>
% 1) Código LuaLatex:
%    Licencia GPL-2.
% 2) Producto en pdf, postscript, etc.:
%    Licencia Creative Commons Recognition Share alike. (CC-BY-SA)

\section{Propiedades}
El sistema \textsf{GObject} contempla el que las instancias puedan contener propiedades. Las propiedades son valores que se almacenan en las instancias que descienden de \textsf{GObject} y, a su vez, estarían disponibles para posibles instancias hijas.
Las propiedades se identifican, y se accede a ellas, mediante sus nombres.

Por ejemplo, \textsf{GtkWindow} tiene las propiedades "\texttt{title}",
"\texttt{default-width}" y "\texttt{default-height}", entre otras. La cadena
"\texttt{title}" es el nombre de una propiedad. El nombre de una propiedad es una cadena
que comienza con una letra seguida de letras, dígitos, guiones ('-') o subrayados
('\_'). Estos dos últimos se utilizan como separadores, pero no se pueden mezclar. El
uso de guiones es más eficiente que el de subrayados. Por ejemplo, "\texttt{value}",
"\texttt{double}" y "\texttt{double-value}" son nombres válidos de propiedades.
"\texttt{-value}" o "\texttt{\_value}" son incorrectos.

A las propiedades se les asigna un tipo. El tipo de la propiedad "\texttt{title}" es una cadena, el de "\texttt{default-width}" y "\texttt{default-height}" es un entero.

Las propiedades se almacenan y se recuperan mediante funciones definidas en
\textsf{GObject}:
\vspace{-\topsep}
\begin{itemize}
  \tightlist
\item Las que más se utilizan para almacenarlas son \texttt{g\_object\_new} y \texttt{g\_object\_set}.
\item Para recuperarlas, la que se usa más frecuentemente es \texttt{g\_object\_get}.
\end{itemize}

A continuación se muestra un ejemplo de \textsf{GtkWindow}, que desciende de
\textsf{GObject}. Mediante la función \texttt{g\_object\_new} se crea una instancia y se
definen sus propiedades:
\begin{lstlisting}[language=C]
GtkWindow *win;
win = g_object_new (GTK_TYPE_WINDOW,
                    "title", "Hola", "default-width", 800, "default-height", 600,
                    NULL);
\end{lstlisting}
El ejemplo anterior crea una instancia de \textsf{GtkWindow} y almacena sus propiedades.

\vspace{-\topsep}
\begin{itemize}
  \tightlist
\item A la propiedad "\texttt{title}" se le asigna "\texttt{Hola}".
\item A "\texttt{default-width}" se le asigna \texttt{800}.
\item A "\texttt{default-height}" se le asigna \texttt{600}.
\item El último parámetro de \texttt{g\_object\_new} es \texttt{NULL} que indica el
  final de la lista de propiedades.
\end{itemize}

Si ya se hubiera creado la instancia de \textsf{GtkWindow} y quisiéramos asignar su
título, se puede usar \texttt{g\_object\_set}:
\begin{lstlisting}[language=C]
GtkWindow *win;
win = g_object_new (GTK_TYPE_WINDOW, NULL);
g_object_set (win, "title", "Adiós", NULL);
\end{lstlisting}

Para recuperar el valor de una propiedad se puede usar \texttt{g\_object\_get}:

\begin{lstlisting}[language=C]
GtkWindow *win;
char *title;
int width, height;

win = g_object_new (GTK_TYPE_WINDOW,
                    "title", "Hello", "default-width", 800, "default-height", 600,
                    NULL);
g_object_get (win,
              "title", &title, "default-width", &width, "default-height", &height,
              NULL);
g_print ("%s, %d, %d\n", title, width, height);
g_free (title);
\end{lstlisting}

En el resto de esta sección se explica cómo se implementan las propiedades de una
instancia que descienda de \textsf{GObject}. La explicación se divide en dos partes:
\vspace{-\topsep}
\begin{itemize}
  \tightlist
\item Registro de una propiedad.
\item Definición de los métodos de clase \texttt{set\_property} y \texttt{get\_property},
  que se relacionan con \texttt{g\_object\_set} y \texttt{g\_object\_get}.
\end{itemize}

\subsection{\textsf{GParamSpec}}
\textsf{GParamSpec} es un objeto fundamental. \textsf{GParamSpec} y \textsf{GObject}
no tienen una relación padre-hijo. \textsf{GParamSpec} lleva la información de los
parámetros. \textsf{"ParamSpec"} es una abreviatura de "\emph{Parameter Specification"}.
Por ejemplo:

\begin{lstlisting}[language=C]
double_property = g_param_spec_double ("value", "val", "Valor doble",
                                       -G_MAXDOUBLE, G_MAXDOUBLE, 0.0,
                                        G_PARAM_READWRITE);
\end{lstlisting}

Esta función crea una instancia de \textsf{GParamSpec}, más precisamente, una instancia
de \textsf{GParamSpecDouble}. Esta última es hija de la primera.

La instancia tiene la siguiente información:
\vspace{-\topsep}
\begin{itemize}
  \tightlist
\item El tipo del valor es \texttt{double}. El sufijo del nombre de la función,
  \texttt{double} en \texttt{g\_param\_spec\_double} lo indica.
\item El nombre es \texttt{"value"}.
\item El apodo es \texttt{"val"}.
\item La descripción es \texttt{"Valor doble"}.
\item El mínimo valor es \texttt{-G\_MAXDOUBLE}, el máximo es \texttt{G\_MAXDOUBLE}.
  Estos valores se describen en el manual de referencia de GLib
  \href{https://docs.gtk.org/glib/types.html#gdouble}{(GLib Reference Manual -- \texttt{G\_MAXDOUBLE} and \texttt{G\_MINDOUBLE})}.
\item El valor por defecto es \texttt{0.0}.
\item \texttt{G\_PARAM\_READWRITE} es una bandera (\emph{flag}). Significa que el
  parámetro es de lectura--escritura.
\item Para más información ver la API de referencia de \textsf{GObject}:
  \vspace{-\topsep}
  \vspace{0.8ex}
  \begin{itemize}
    \tightlist
  \item \href{https://docs.gtk.org/gobject/index.html#classes}
    {\texttt{GParamSpec} and its subclasses}.
  \item \href{https://docs.gtk.org/gobject/index.html#functions}
    {\texttt{g\_paramspec\_double} and similar functions}.
  \item \href{https://docs.gtk.org/gobject/struct.Value.html}
    {\texttt{GValue}}.
  \end{itemize}
\end{itemize}

\textsf{GParamSpec} se usa para registrar propiedades de instancias \textsf{GObject}
o descendientes de esta. Este código se extrae del ejemplo xx:
\begin{lstlisting}[language=C]
#define PROP_DOUBLE 1
static GParamSpec *double_property = NULL;

static void
t_double_class_init (TDoubleClass *class) {
  GObjectClass *gobject_class = G_OBJECT_CLASS (class);

  gobject_class->set_property = t_double_set_property;
  gobject_class->get_property = t_double_get_property;
  double_property = g_param_spec_double ("value", "val", "Double value", -G_MAXDOUBLE, G_MAXDOUBLE, 0.0, G_PARAM_READWRITE);
  g_object_class_install_property (gobject_class, PROP_DOUBLE, double_property);
}
\end{lstlisting}

La variable \texttt{double\_property} es estática. Esta variable es una instancia de \texttt{GParamSpec}.

La función \texttt{g\_object\_class\_install\_property} instala una propiedad.
Se debe ejecutar después de que se hayan sobrescrito los métodos \texttt{set\_property}
y \texttt{get\_property}. Estos métodos se explicarán más adelante. Los argumentos
son la clase \texttt{TDoubleClass}, el identificador de la propiedad
\texttt{PROP\_DOUBLE} y la instancia \texttt{GParamSpec}. El identificador de la
propiedad se usa para identificar la propiedad en \textsf{tdouble.c} y es un entero
positivo.

\subsection{Sobrescritura de los métodos de clase \texttt{set\_property} y
  \texttt{get\_property}}
Los valores de las propiedades pueden diferir de una instancia a otra. Por lo tanto,
el valor se almacena en cada una de las instancias, independientemente del resto de
las demás.

En la función \texttt{g\_object\_set} se presenta un valor como argumento para
almacenarlo. Pero, ¿cómo sabe esta función en qué instancia se almacena el valor?
Esto se compila antes de que se haga el cree la instancia. Así, no sabe en absoluto
en qué instancia se almacenará. Esa parte se tendrá que programar cuando se
diseñe el objeto a través de la sobrescritura.

La función \texttt{g\_object\_set} primero comprueba la propiedad y el valor, después
crea un valor genérico \texttt{GValue} para la propiedad. Finalmente, llama a una
función cuyo puntero se pasa por \texttt{set\_property} en la clase. Observar el
siguiente diagrama.




  
%%% Local Variables:
%%% coding: utf-8
%%% mode: latex
%%% TeX-engine: luatex
%%% TeX-master: "../GObject_tutorial_es.tex"
%%% End:

% LaTeX-command: "lualatex --shell-escape"
