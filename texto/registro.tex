% registro.tex
%
% Copyright (C) 2024 José A. Navarro Ramón <janr.devel@gmail.com>
% 1) Código LuaLatex:
%    Licencia GPL-2.
% 2) Producto en pdf, postscript, etc.:
%    Licencia Creative Commons Recognition Share alike. (CC-BY-SA)

\section{Registro}
\subsection{Sistema de tipos y registro de clases derivadas de \textsf{GObject}} \label{subsec:sistematipos_y_registro}
\subsubsection{Introducción}
\textsf{GObject} es un objeto básico. Es muy simple y no se puede hacer mucho con él, excepto crear
clases derivadas e instanciar nuevos tipos de objetos.
En realidad esta es la característica más importante de \textsf{GObject}.
Aquí se describe como definir clases derivadas de \textsf{GObject}.

En esta sección se crearán objetos que representan números reales. Este ejemplo no será muy útil, porque
el lenguaje C ya tiene un tipo \emph{double} que representa números reales. Sin embargo, será útil para
conocer la técnica para definir clases y objetos derivados de \textsf{GObject}.

\subsubsection{Convenio de nombres}
Primero se debe conocer el convenio de nombres que se usará. El nombre de un objeto tiene dos partes:
\begin{enumerate}
  \tightlist
\item Un \href{https://es.wikipedia.org/wiki/Espacio_de_nombres}{espacio de nombres}
  (\href{https://en.wikipedia.org/wiki/Namespace}{\textit{namespace}}).
\item Un nombre propiamente dicho, que se considera incluido en el espacio de nombres.
\end{enumerate}

Por ejemplo, \textsf{GObject} consiste en un espacio de nombres ``\textsf{G}'' y un nombre ``\textsf{Object}''.
\textsf{GtkWidget} tiene un espacio de nombres ``\textsf{Gtk}'' y el nombre ``\textsf{Widget}''.
Para el objeto de números reales que pondremos como ejemplo, se decide poner ``\textsf{T}'' como espacio
de nombres y ``\textsf{Doble}'' como nombre. Así, estos objetos se identificarán como \textsf{TDoble}.
Además serán objetos que derivan de \textsf{GObject}, representan números reales y el tipo del número
será \textit{double}.

\subsubsection{Tipo}
Cuando decimos ``tipo'', nos podemos referir a dos conceptos diferentes, aunque relacionados.
Por un lado, el tipo del sistema de objetos y, por otro, el tipo del lenguaje C. Por ejemplo, \textsf{GObject} es
un tipo del sistema de tipos. Pero, ``char'', o `double'', por otro lado son tipos del lenguaje C.
Cuando el significado de ``tipo'' quede claro por el contexto, diremos sencillamente ``tipo''; en caso contrario
diremos ``tipo de C'' o ``tipo del sistema''.

\subsubsection{Definición de \textsf{TDobleClass} y de \textsf{TDoble}}
  Nuestro objeto \textsf{TDoble} pertenecerá a la clase \textsf{TDobleClass}. La estructura en C de esta última es
  \begin{lstlisting}[language=C]
    typedef struct _TDobleClass TDobleClass;
    struct _TDobleClass {
      GObjectClass parent_class;
    };
  \end{lstlisting}

  \texttt{\_TDobleClass} es el nombre de una estructura en C y \texttt{TDobleClass} es \texttt{struct \_TDobleClass}.
  Así, \textsf{TDobleClass} es un tipo de C recién definido.
  En el código anterior, se usa \texttt{typedef} para definir un tipo de clase. El primer miembro de la estructura
  debe ser la estructura de la clase padre. \textsf{TDobleClass} no necesita más información. El tipo C de una
  instancia de \textsf{TDoble} es \texttt{TDoble}.
  \begin{lstlisting}[language=C]
    typedef struct _TDoble TDoble;
    struct _TDoble {
      GObject parent;
      double valor;
    };
  \end{lstlisting}
  
  Esto es similar a la estructura de la clase. Primero se usa \texttt{typedef} para definir el tipo en C de una
  instancia de la clase. El primer miembro de la estructura debe ser la estructura de la instancia padre.
  \textsf{TDoble} tiene su propio miembro ``\texttt{valor}'', que es el valor de las instancias de \textsf{TDoble}.
  La convención de código mostrada en los dos listados anteriores debe respetarse siempre.

  \subsubsection{Procedimiento para crear un objeto descendiente de \textsf{GObject}}
  Para crear un tipo \text{TDoble} hay que:
  \begin{enumerate}
    \tightlist
  \item Registro del tipo en C \texttt{TDoble} en el sistema de tipos.
  \item El sistema de tipos asigna memoria para \textsf{TDobleClass} y \textsf{TDoble}.
  \item Inicialización de \textsf{TDobleClass}.
  \item Inicialización de \textsf{TDoble}.
  \end{enumerate}

  


%%% Local Variables:
%%% coding: utf-8
%%% mode: latex
%%% TeX-engine: luatex
%%% TeX-master: "../GObject_tutorial_es.tex"
%%% End:

% LaTeX-command: "lualatex --shell-escape"
