% registro.tex
%
% Copyright (C) 2024 José A. Navarro Ramón <janr.devel@gmail.com>
% 1) Código LuaLatex:
%    Licencia GPL-2.
% 2) Producto en pdf, postscript, etc.:
%    Licencia Creative Commons Recognition Share alike. (CC-BY-SA)

\section{Registro}
\subsection{Sistema de tipos y registro de clases} \label{subsec:sistematipos_y_registro}
\subsubsection{Introducción}
\textsf{GObject} es un objeto básico. Es muy simple y no se puede hacer mucho con él, excepto crear
clases derivadas e instanciar nuevos tipos de objetos.
En realidad esta es la característica más importante de \textsf{GObject}.
Aquí se describe como definir clases derivadas de \textsf{GObject}.

En esta sección se crearán objetos que representan números reales. Este ejemplo no será muy útil, porque
el lenguaje C ya tiene un tipo \emph{double} que representa números reales. Sin embargo, será útil para
conocer la técnica para definir clases y objetos derivados de \textsf{GObject}.

\subsubsection{Convenio de nombres}
Primero se debe conocer el convenio de nombres que se usará. El nombre de un objeto tiene dos partes:
\begin{enumerate}
  \tightlist
\item Un \href{https://es.wikipedia.org/wiki/Espacio_de_nombres}{espacio de nombres}
  (\href{https://en.wikipedia.org/wiki/Namespace}{\textit{namespace}}).
\item Un nombre propiamente dicho, que se considera incluido en el espacio de nombres.
\end{enumerate}

Por ejemplo, \textsf{GObject} consiste en un espacio de nombres ``\textsf{G}'' y un nombre ``\textsf{Object}''.
\textsf{GtkWidget} tiene un espacio de nombres ``\textsf{Gtk}'' y el nombre ``\textsf{Widget}''.
Para el objeto de números reales que pondremos como ejemplo, se decide poner ``\textsf{T}'' como espacio
de nombres y ``\textsf{Doble}'' como nombre. Así, estos objetos se identificarán como \textsf{TDoble}.
Además serán objetos que derivan de \textsf{GObject}, representan números reales y el tipo del número
será \textit{double}.

\begin{figure}[ht]
  \centering
  \def\scl{1}
  \begin{tikzpicture}[%
    scale=\scl,
    every node/.style={font=\LARGE\sffamily, minimum size= 8ex},
     background/.style={
      line width=\bgborderwidth,
      draw=\bgbordercolor,
      fill=\bgcolor,
    },
   ]
   % COORDENADAS
   % - 
   % DIBUJO
   % Convenio de nombres de objetos
   \node[draw] (namespace)
   {\begin{tabular}{c} \textcolor{gray}{Espacio de nombres} \\ T \end{tabular}};
   \node [draw, right=1em of namespace] (name)
   {\begin{tabular}{c} \textcolor{gray}{Nombre} \\ Doble \end{tabular}};
   \node [right=1em of name] (tdoble) {$\equiv$ TDoble}; 
   % Fondo amarillo
   \def\margensup{18}
   \def\margeninf{12}
   \def\margenlateralizdo{18}
   \def\margenlateraldcho{12}
   \coordinate (limsupdcha) at ($(tdoble.north east)+(\margenlateraldcho pt, \margensup pt)$);
   \coordinate (liminfizda) at ($(namespace.south west)-(\margenlateralizdo pt, \margeninf pt)$);
   \begin{scope}[on background layer]
     \node [background, fit= (liminfizda) (limsupdcha)] {};
   \end{scope}
\end{tikzpicture}
\caption{Nombre del objeto \textsf{TDoble}.}
\label{fig:nombre-objetos}
\end{figure}


\subsubsection{Tipo}
Cuando decimos ``tipo'', nos podemos referir a dos conceptos diferentes, aunque relacionados.
Por un lado, el tipo del sistema de objetos y, por otro, el tipo del lenguaje C. Por ejemplo, \textsf{GObject} es
un tipo del sistema de tipos. Pero, ``char'', o `double'', por otro lado son tipos del lenguaje C.
Cuando el significado de ``tipo'' quede claro por el contexto, diremos sencillamente ``tipo''; en caso contrario
diremos ``tipo de C'' o ``tipo del sistema''.

\subsubsection{Definición de \textsf{TDobleClass} y de \textsf{TDoble}}
  Nuestro objeto \textsf{TDoble} pertenecerá a la clase \textsf{TDobleClass}. La estructura en C de esta última es
  \begin{lstlisting}[language=C]
    typedef struct _TDobleClass TDobleClass;
    struct _TDobleClass {
      GObjectClass parent_class;
    };
  \end{lstlisting}

  \texttt{\_TDobleClass} es el nombre de una estructura en C y \texttt{TDobleClass} es \texttt{struct \_TDobleClass}.
  Así, \textsf{TDobleClass} es un tipo de C recién definido.
  En el código anterior, se usa \texttt{typedef} para definir un tipo de clase. El primer miembro de la estructura
  debe ser la estructura de la clase padre. \textsf{TDobleClass} no necesita más información. El tipo C de una
  instancia de \textsf{TDoble} es \texttt{TDoble}.
  \begin{lstlisting}[language=C]
    typedef struct _TDoble TDoble;
    struct _TDoble {
      GObject parent;
      double valor;
    };
  \end{lstlisting}
  
  Esto es similar a la estructura de la clase. Primero se usa \texttt{typedef} para definir el tipo en C de una
  instancia de la clase. El primer miembro de la estructura debe ser la estructura de la instancia padre.
  \textsf{TDoble} tiene su propio miembro ``\texttt{valor}'', que es el valor de las instancias de \textsf{TDoble}.
  La convención de código mostrada en los dos listados anteriores debe respetarse siempre.

  \subsubsection{Procedimiento para crear un objeto descendiente de \textsf{GObject}}
  Para crear un tipo \text{TDoble} hay que:
  \begin{enumerate}
    \tightlist
  \item Registro del tipo en C \texttt{TDoble} en el sistema de tipos.
  \item El sistema de tipos asigna memoria para \textsf{TDobleClass} y \textsf{TDoble}.
  \item Inicialización de \textsf{TDobleClass}.
  \item Inicialización de \textsf{TDoble}.
  \end{enumerate}

%  \begin{figure}[ht]
%  \centering
%  \def\scl{1}
%  \newcommand{\fondoTipoC}{green!10}
%  \newcommand{\fondoTipoSistema}{green!20}
%  \begin{tikzpicture}[%
%    scale=\scl,
%    every node/.style={font=\large\sffamily},
%    nube/.style={fill=yellow!75},
%    tipoC/.style={fill=\fondoTipoC, minimum size=8ex, inner sep=1em},
%    tipoSistema/.style={fill=\fondoTipoSistema, minimum size=8ex, inner sep=1em},
%    flecha/.style={line width=1.6pt, -Stealth[round]},  background/.style={
%      line width=\bgborderwidth,
%      draw=\bgbordercolor,
%      fill=\bgcolor,
%    },
%   ]
%   % COORDENADAS
%   % - 
%   % DIBUJO
%   % Proceso de registro de una clase
%   % Type System
%   \node[nube, cloud, draw, aspect=2] (typesystem) {Sistema de tipos};
%   \node[nube, cloud, draw, aspect=2, right=12em of typesystem] (memory)  {Memoria};
%   \node[tipoC, draw, above=4em of typesystem] (tdobletype)
%   {\begin{tabular}{c}TDoble\\\textcolor{black!70}{(Tipo C)}\end{tabular}};
%   \node[tipoSistema, draw, below=6em of memory] (tdobleclass)
%   {\begin{tabular}{c} Clase TDobleClass \\ \textcolor{black!70}{(Tipo sistema)} \end{tabular}};
%   \node[tipoSistema, draw, left=11em of tdobleclass] (tdobleobject)
%   {\begin{tabular}{c} Objeto TDoble \\ \textcolor{black!70}{(Tipo sistema)} \end{tabular}};
%   % Flechas
%   \draw[flecha] (tdobletype.south) --
%   node[left] {$\color{blue}{\mbfone}$} node[right]  {\textcolor{blue}{Registro}}
%   (typesystem.north);
%   \draw[flecha] (typesystem.east) -- (memory.west);
%   \draw[flecha] (memory.south) -- (tdobleclass.north);
%   \draw[flecha] (tdobleclass.west) -- (tdobleobject.east);
%   
%   %   % Fondo amarillo
%%   \def\margensup{18}
%%   \def\margeninf{12}
%%   \def\margenlateralizdo{18}
%%   \def\margenlateraldcho{12}
%%   \coordinate (limsupdcha) at ($(tdoble.north east)+(\margenlateraldcho pt, \margensup pt)$);
%%   \coordinate (liminfizda) at ($(namespace.south west)-(\margenlateralizdo pt, \margeninf pt)$);
%%   \begin{scope}[on background layer]
%%     \node [background, fit= (liminfizda) (limsupdcha)] {};
%%   \end{scope}
%\end{tikzpicture}
%\caption{Proceso de registro del objeto \textsf{TDoble}..}
%\label{fig:proceso-registro}
%\end{figure}

  \begin{figure}[ht]
  \centering
  \def\scl{1}
  \newcommand{\fondoTipoC}{green!20}
  \newcommand{\fondoTipoSistema}{green!25}
  \begin{tikzpicture}[%
    scale=\scl,
    every node/.style={font=\normalsize\sffamily},
    nubeSistemaTipos/.style={fill=yellow!75},
    nubeMemoria/.style={fill=black!2},
    tipoC/.style={fill=\fondoTipoC, font=\ttfamily, minimum size=8ex, inner sep=1em},
    tipoSistema/.style={fill=\fondoTipoSistema, minimum size=8ex, inner sep=1em},
    flecha/.style={line width=1.6pt, -Stealth[round]},  background/.style={
      line width=\bgborderwidth,
      draw=\bgbordercolor,
      fill=\bgcolor,
    },
   ]
   % COORDENADAS
   % - 
   % DIBUJO
   % Proceso de registro de una clase
   % Type System
   \node[nubeSistemaTipos, cloud, draw, aspect=2] (typesystem) {Sistema de tipos};
   % Memoria
   \node[nubeMemoria, cloud, draw, aspect=2, right=12em of typesystem] (memory)
   {\begin{tabular}{c} Memoria \\ (TDobleClass y TDoble) \end{tabular}};
   % TDoble (Tipo C)
   \node[tipoC, draw, above=8ex of typesystem] (tdobleCtype)
   {\begin{tabular}{c}TDoble\\\textcolor{black!70}{(Tipo C)}\end{tabular}};
   % Objeto TDoble
   \node[tipoSistema, draw, below=8ex of memory] (tdobleobject)
   {\begin{tabular}{c} Objeto TDoble \\ \textcolor{black!70}{(Tipo sistema)} \end{tabular}};
   % Clase TDobleClass
   \node[tipoSistema, draw, left=12em of tdobleobject] (tdobleclass)
   {\begin{tabular}{c} Clase TDobleClass \\ \textcolor{black!70}{(Tipo sistema)} \end{tabular}};
   % Flechas
   % TDoble (Tipo C) -> Sistema Tipos
   \draw[flecha] (tdobleCtype.south) --
   node[left] {\small $\mbfone$} node[right]  {\small Registro}
   (typesystem.north);
   % Sistema tipos -> memoria
   \draw[flecha] (typesystem.east) --
   node[below]{\small Asignación de memoria}
   node[above]{\small $\mbftwo$}
   (memory.west);
   % Memoria -> Clase TDobleClass (Tipo sistema)
   \draw[flecha] (memory.225) --
   node[above right, sloped]{\small $\mbfthree$}
   node[below right=0.2ex and -4ex, sloped]{\small Inicialización}
   (tdobleclass.north);
   % Memoria -> Objeto TDoble (Tipo sistema)
   \draw[flecha] (memory.south) --
   node[left] {\small $\mbffour$}
   node[right] {\small Inicialización}
   (tdobleobject.north);
   
% Fondo amarillo
   \def\margen{16}
   \coordinate (limizda) at ($(typesystem.west)-(\margen pt, 0pt)$);
   \coordinate(limdcha) at ($(memory.east)+(\margen pt, 0pt)$);
   \coordinate(liminf) at ($(tdobleobject.south)-(0pt, \margen pt)$);
   \coordinate(limsup) at ($(tdobleCtype.north)+(0pt, \margen pt)$);
   \begin{scope}[on background layer]
     \node [background, fit= (liminf) (limsup) (limizda) (limdcha)] {};
   \end{scope}
\end{tikzpicture}
\caption{Proceso de registro del objeto \textsf{TDoble}..}
\label{fig:proceso-registro}
\end{figure}

  


%%% Local Variables:
%%% coding: utf-8
%%% mode: latex
%%% TeX-engine: luatex
%%% TeX-master: "../GObject_tutorial_es.tex"
%%% End:

% LaTeX-command: "lualatex --shell-escape"
