% tipos.tex
%
% Copyright (C) 2024 José A. Navarro Ramón <janr.devel@gmail.com>
% 1) Código LuaLatex:
%    Licencia GPL-2.
% 2) Producto en pdf, postscript, etc.:
%    Licencia Creative Commons Recognition Share alike. (CC-BY-SA)

\section{Sistema de tipos}
%\subsection{Sistema de tipos} \label{subsec:sistematipos}
\subsection{Introducción}
\textsf{GObject} es un objeto básico. Es muy simple y no se puede hacer mucho con él, excepto crear
clases derivadas e instanciar nuevos tipos de objetos.
En realidad esta es la característica más importante de \textsf{GObject}.

Más adelante se describirá como definir clases derivadas de \textsf{GObject}.
En particular se creará una clase de objeto  que representa un número real.
Este ejemplo no es muy útil porque el lenguaje C ya tiene un tipo \texttt{double} que representa
números en coma flotante, pero servirá para conocer la técnica para definir clases y objetos derivados de \textsf{GObject}.

\subsection{Convenio de nombres}
Primero se debe conocer el convenio de nombres que usa el sistema de tipos.
El nombre de un objeto tiene dos partes:
\begin{enumerate}
  \tightlist
\item Un \href{https://es.wikipedia.org/wiki/Espacio_de_nombres}{espacio de nombres}
  (\href{https://en.wikipedia.org/wiki/Namespace}{\textit{namespace}}).
\item Un nombre propiamente dicho, que pertenece al espacio de nombres.
\end{enumerate}

Por ejemplo, \textsf{GObject} consiste en un espacio de nombres ``\textsf{G}'' y un nombre ``\textsf{Object}''.
\textsf{GtkWidget} tiene un espacio de nombres ``\textsf{Gtk}'' y el nombre ``\textsf{Widget}''.

Para el objeto de números reales de nuestro ejemplo, se decide poner ``\textsf{T}'' como espacio
de nombres y ``\textsf{Doble}'' como nombre. Así, estos objetos se identificarán como \textsf{TDoble},
derivarán de \textsf{GObject}, representan números reales y el tipo del número en
lenguaje C será \texttt{double}. Diremos que \textsf{TDoble} contiene un parámetro, que llamaremos
\texttt{valor} que se  define como \texttt{double valor} en lenguaje C.

\begin{figure}[ht]
  \centering
  \def\scl{1.4}
  \begin{tikzpicture}[%
    scale=\scl,
    every node/.style={font=\LARGE\sffamily, minimum size= 8ex},
     background/.style={
      line width=\bgborderwidth,
      draw=\bgbordercolor,
      fill=\bgcolor,
    },
   ]
   % COORDENADAS
   % - 
   % DIBUJO
   % Convenio de nombres de objetos
   \node[draw] (namespace)
   {\begin{tabular}{c} \textcolor{gray}{Espacio de nombres} \\ T \end{tabular}};
   \node [draw, right=1em of namespace] (name)
   {\begin{tabular}{c} \textcolor{gray}{Nombre} \\ Doble \end{tabular}};
   \node [right=1em of name] (tdoble) {$\equiv$ TDoble}; 
   % Fondo amarillo
   \def\margensup{18}
   \def\margeninf{12}
   \def\margenlateralizdo{18}
   \def\margenlateraldcho{12}
   \coordinate (limsupdcha) at ($(tdoble.north east)+(\margenlateraldcho pt, \margensup pt)$);
   \coordinate (liminfizda) at ($(namespace.south west)-(\margenlateralizdo pt, \margeninf pt)$);
   \begin{scope}[on background layer]
     \node [background, fit= (liminfizda) (limsupdcha)] {};
   \end{scope}
\end{tikzpicture}
\caption{Nombre del objeto \textsf{TDoble}.}
\label{fig:nombre-objetos}
\end{figure}

\subsection{Tipo}
Cuando decimos ``tipo'', nos podemos referir a dos conceptos diferentes, aunque relacionados.
Por un lado, el tipo del sistema de objetos de \textsf{GObject} (que es un subtipo de \texttt{GType}) y,
por otro, el tipo del lenguaje C. Por ejemplo, \textsf{GObject} es un tipo del sistema de tipos \texttt{GType}.
Pero \texttt{char}, o \texttt{double}, por otro lado, son tipos del lenguaje C.
Cuando el significado de ``tipo'' quede claro por el contexto, diremos sencillamente ``tipo''; en caso contrario
diremos ``tipo de C'' o ``tipo del sistema''.

\subsubsection{Definición de \textsf{TDobleClass} y de \textsf{TDoble}}
Empezaremos definiendo las estructuras de clase e instancia de nuestro objeto.
Una instancia \textsf{TDoble} pertenecerá a la clase \textsf{TDobleClass}. La estructura en C de esta última es:
\begin{lstlisting}[language=C]
  typedef struct _TDobleClass TDobleClass;
  struct _TDobleClass {
    GObjectClass parent_class;
  };
\end{lstlisting}

\texttt{\_TDobleClass} es el nombre de una estructura en C y \texttt{TDobleClass} es el tipo C que equivale
a \texttt{struct \_TDobleClass}. De esta manera, hemos definido el tipo C, \texttt{TDobleClass}.
En el código anterior, se usa \texttt{typedef} para definir el tipo C de la clase. El primer miembro de la estructura
debe ser la estructura de la clase padre. \textsf{TDobleClass} no necesita más información porque no es una
clase abstracta.

\newpage
El tipo C de una instancia de \textsf{TDoble} es \texttt{TDoble}:
\begin{lstlisting}[language=C]
  typedef struct _TDoble TDoble;
  struct _TDoble {
    GObject parent;
    double valor;
  };
\end{lstlisting}
  
El código es similar a la estructura de la clase. Primero se usa \texttt{typedef} para definir el tipo en C de una
instancia de la clase. El primer miembro de la estructura debe ser la estructura de la instancia padre.
\textsf{TDoble} tiene su propio miembro o parámetro, \texttt{valor}, que es el número en coma flotante
de las instancias de \textsf{TDoble}.
La convención de código mostrada en los dos listados anteriores debe respetarse siempre.




%%% Local Variables:
%%% coding: utf-8
%%% mode: latex
%%% TeX-engine: luatex
%%% TeX-master: "../GObject_tutorial_es.tex"
%%% End:

% LaTeX-command: "lualatex --shell-escape"
