% compilacion.tex
%
% Copyright (C) 2024 José A. Navarro Ramón <janr.devel@gmail.com>
% 1) Código LuaLatex:
%    Licencia GPL-2.
% 2) Producto en pdf, postscript, etc.:
%    Licencia Creative Commons Recognition Share alike. (CC-BY-SA)

\section{Compilación de programas}\label{sec:compilacion}

\begin{enumerate}
  \item Para compilar los programas que nos interesan, primero hay que entrar en el directorio correspondiente:
\begin{itemize}
  \tightlist
\item '\textbf{\texttt{\$ cd prog/ej0105}}':
  para compilar '\textsf{ejemplo01}', '\textsf{ejemplo02}', '\textsf{ejemplo03}', '\textsf{ejemplo04}'
  y '\textsf{ejemplo05}'.
\item '\textbf{\texttt{\$ cd prog/ej06}}': para compilar '\texttt{ejemplo06}'.
\item '\textbf{\texttt{\$ cd prog/operaciones}}': para compilar '\texttt{operaciones}'.
\end{itemize}

\item A continuación, se elige el sistema que se prefiere para construir el/los ejecutables.
  En este tutorial consideramos los siguientes:
  \begin{itemize}
    \tightlist
  \item Para utilizar \textbf{cmake}, se puede utilizar \emph{ninja} o \emph{make} como las herramientas
    que llevarán a cabo la compilación.
    \begin{itemize}
      \item Si queremos compilar con \emph{ninja}:
        \begin{lstlisting}[language=bash]
          $ cd cmake
          $ cmake -B build -G Ninja
          $ ninja -C build
        \end{lstlisting}
      \item Si queremos compilar con \emph{make}:
        \begin{lstlisting}[language=bash]
          $ cd cmake
          $ cmake -B build -G "Unix Makefiles"
          $ cd build
          $ make
        \end{lstlisting}
    \end{itemize}
    Los ejecutables se encuentran en el directorio elegido (en este caso 'build').
    El comando \texttt{cmake ...} da la orden de compilar con códigos de depuración (\textsf{Debug}).

    Se podría elegir otra forma de compilación, añadiendo al final de '\texttt{cmake ...}'
    el argumento \texttt{-DCMAKE\_BUILD\_TYPE=} con los posibles valores a la derecha del signo 'igual que':
    \begin{itemize}
      \tightlist
    \item '\texttt{-DCMAKE\_BUILD\_TYPE=Release}': Sin códigos de depuración.
    \item '\texttt{-DCMAKE\_BUILD\_TYPE=Debug}': Con códigos de depuración.
    \item '\texttt{-DCMAKE\_BUILD\_TYPE=RelWithDebInfo}': Release con códigos de depuración.
    \item '\texttt{-DCMAKE\_BUILD\_TYPE=MinSizeRel}': Release de tamaño mínimo (\texttt{strip}).
    \end{itemize}
  \item Para utilizar \textbf{make}, se teclea
    Ahora se ejecutan los siguientes comandos:
    \begin{lstlisting}[language=bash]
      $ cd make
      $ make
    \end{lstlisting}
    Los ejecutables se encuentran en el directorio actual.
  \item Para utilizar \textbf{meson}, se teclea:
    \begin{lstlisting}[language=bash]
    $ cd meson
    $ meson setup build
    $ ninja -C build
  \end{lstlisting}
  Los ejecutables se encuentran en el directorio elegido (en este caso 'build').
  El comando \texttt{meson ...} da la orden de compilar con códigos de depuración (\textsf{Debug}).

  Se podría elegir otra forma de compilación, añadiendo al final del comando '\texttt{meson ...}'
  el argumento \texttt{--buildtype=} con los posibles valores a la derecha del signo 'igual que':
  \begin{itemize}
    \tightlist
  \item '\texttt{--buildtype=release}': Sin códigos de depuración.
  \item '\texttt{--buildtype=debug}': Con códigos de depuración.
  \end{itemize}

\item Para borrar los ficheros que se han creado, se teclea (en cualquiera de los tres tipos de compilación):
  \begin{lstlisting}[language=bash]
    $ make clean
  \end{lstlisting}
  \end{itemize}
\end{enumerate}





%%% Local Variables:
%%% coding: utf-8
%%% mode: latex
%%% TeX-engine: luatex
%%% TeX-master: "../GObject_tutorial_es.tex"
%%% End:

% LaTeX-command: "lualatex --shell-escape"
